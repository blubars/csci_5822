\documentclass[letterpaper, 10pt]{article}
%\usepackage{pgf} % import pgf plots from matplotlib
\usepackage{amsmath,amsthm, amssymb}
\usepackage{bm}
%\usepackage{mathmode}
\usepackage{listings}
\usepackage{graphicx}

%opening
\title{CSCI 5822: Homework 9}
\author{Brian Lubars}
\date{5/02/18}

\addtolength{\oddsidemargin}{-.875in}
\addtolength{\evensidemargin}{-.875in}
\addtolength{\textwidth}{1.75in}
%\addtolength{\topmargin}{-.875in}
\addtolength{\topmargin}{-.875in}
\addtolength{\textheight}{2in}

\begin{document}

\maketitle
\section{Part 1}
\subsection{Part 1a}
Implement a function over which we will optimize. This function takes an input vector and outputs a function value. 
The function is an inverted paraboloid with two multivariate Gaussians added to it:

\section{Part 2}
\subsection{Part 2a}

\begin{figure}[h!]
  \centering
  \includegraphics[width=15cm]{ds_mode_0.png}
  \label{1}
  \caption{Linear Dynamical System}
\end{figure}

\subsection{Part 2b}

\begin{figure}[h!]
  \centering
  \includegraphics[width=15cm]{ds_mixed.png}
  \label{2}
  \caption{Switched Mode Linear Dynamical System}
\end{figure}

\begin{table}[]
\centering
\caption{Parameters for linear dynamical system}
\label{my-label}
\begin{tabular}{lll}
\textbf{mode} & \textbf{alpha1} & \textbf{alpha2} \\
1             & 0.5             & 0.5             \\
2             & 0.07            & 0.9             \\
3             & 0.1             & 0.1            
\end{tabular}
\end{table}

\begin{table}[]
\centering
\caption{Counts and observed transition probabilities}
\label{my-label}
\begin{tabular}{llll}
\textbf{count (P)} & \textbf{mode 1} & \textbf{mode 2} & \textbf{mode 3} \\
\textbf{mode 1}    & 746 (0.497)     & 3 (.002)        & 4 (.003)        \\
\textbf{mode 2}    & 3 (.002)        & 536 (.357)      & 3 (.002)        \\
\textbf{mode 3}    & 4 (.003)        & 3 (.002)        & 199 (.133)     
\end{tabular}
\end{table}

A state-space model assumes there is a hidden state defined by a continuous vector X, and an observation of the state defined by a vector Z. Some arbitrary observation function maps X to Z. A linear state model assumes that the mapping from $X_{t-1}$ to $X_t$ and $X_t$ to $Z_t$ is linear!

A Kalman Filter is a linear state-space model with Gaussian noise on the state $X$ and the observation $Z$. The model works as follows.

The linear dynamics of a new time step of the state space are modeled by thedifference equation (linear stochastic difference from the previous state): 
$$ x_k = F_k x_{k-1} + B_k u_k+ w_k $$
... where $x_k$ is the new state, $F_k$ is the state-transition matrix and $w_k \sim \mathcal{N}(0, Q_k)$ is the process noise drawn from a normal distribution, and $B_k u_k$ is the models the effects of the inputs on the system (if they are known). 

The observation at each time step modeled by the equation: 
$$ z_k = H_k x_k + v_k $$
... where $H_k$ models the linear function mapping the state $x_k$ to the observation $z_k$ and $v_k \sim \mathcal{N}(0,R_k)$ is observation noise.

The Kalman Filter, then, makes a posterior estimate of the new state based on the predictions of the previous position, dynamics, and noise.

At each step, we calculate the residual based on the new observation, calculate the residual covariance, and calculate the optimal kalman gain based on the residual covariance! Finally, we update the predicted $x_k$.

\end{document}

